% redefine \mathrm to be a \protected macro rather than a `robust' command
\protected\edef\mathrm{%
  \unexpanded\expandafter\expandafter\expandafter{%
    \csname mathrm \endcsname
  }%
}

\newcommand{\numOfFiles}{11}
\newcommand{\figHeight}{8cm}
\newcommand{\figWidth}{8cm}

\newcounter{cnt}
\newcommand\textlist{}
\newcommand\settext[2]{%
  \csdef{text#1}{#2}}
\newcommand\addtext[1]{%
  \stepcounter{cnt}%
  \csdef{text\thecnt}{#1}}
\newcommand\gettext[1]{%
  \csuse{text#1}}
  
\newcounter{savePage}
  
%% Adds a possibility of setting the cycle position %%%%%%%%%%%%%%%%%%%%
\makeatletter
\def\pgfplots@getautoplotspec into#1{%
    \begingroup
    \let#1=\pgfutil@empty
    \pgfkeysgetvalue{/pgfplots/cycle multi list/@dim}\pgfplots@cycle@dim
    %
    \let\pgfplots@listindex=\pgfplots@numplots
    %%% Start new code
    \pgfkeysgetvalue{/pgfplots/cycle list set}\pgfplots@listindex@set
    \ifx\pgfplots@listindex@set\pgfutil@empty
    \else 
        \c@pgf@counta=\pgfplots@listindex
        \c@pgf@countb=\pgfplots@listindex@set
        \advance\c@pgf@countb by -\c@pgf@counta
        \globaldefs=1\relax
        \edef\setshift{%
            \noexpand\pgfkeys{
                /pgfplots/cycle list shift=\the\c@pgf@countb,
                /pgfplots/cycle list set=
            }
        }%
        \setshift%
    \fi
    %%% End new code    
    \pgfkeysgetvalue{/pgfplots/cycle list shift}\pgfplots@listindex@shift
    \ifx\pgfplots@listindex@shift\pgfutil@empty
    \else
        \c@pgf@counta=\pgfplots@listindex\relax
        \advance\c@pgf@counta by\pgfplots@listindex@shift\relax
        \ifnum\c@pgf@counta<0
            \c@pgf@counta=-\c@pgf@counta
        \fi
        \edef\pgfplots@listindex{\the\c@pgf@counta}%
    \fi
    \ifnum\pgfplots@cycle@dim>0
        % use the 'cycle multi list' feature.
        %
        % it employs a scalar -> multiindex map like
        % void fromScalar( size_t d, size_t scalar, size_t* Iout, const size_t* N )
        % {
        %   size_t ret=scalar;
        %   for( int i = d-1; i>=0; --i ) {
        %       Iout[i] = ret % N[i];
        %       ret /= N[i];
        %   }
        % }
        % to get the different indices into the cycle lists.
        %-------------------------------------------------- 
        \c@pgf@counta=\pgfplots@cycle@dim\relax
        \c@pgf@countb=\pgfplots@listindex\relax
        \advance\c@pgf@counta by-1
        \pgfplotsloop{%
            \ifnum\c@pgf@counta<0
                \pgfplotsloopcontinuefalse
            \else
                \pgfplotsloopcontinuetrue
            \fi
        }{%
            \pgfkeysgetvalue{/pgfplots/cycle multi list/@N\the\c@pgf@counta}\pgfplots@cycle@N
            % compute list index:
            \pgfplotsmathmodint{\c@pgf@countb}{\pgfplots@cycle@N}%
            \divide\c@pgf@countb by \pgfplots@cycle@N\relax
            %
            \expandafter\pgfplots@getautoplotspec@
                \csname pgfp@cyclist@/pgfplots/cycle multi list/@list\the\c@pgf@counta @\endcsname
                {\pgfplots@cycle@N}%
                {\pgfmathresult}%
            \t@pgfplots@toka=\expandafter{#1,}%
            \t@pgfplots@tokb=\expandafter{\pgfplotsretval}%
            \edef#1{\the\t@pgfplots@toka\the\t@pgfplots@tokb}%
            \advance\c@pgf@counta by-1
        }%
    \else
        % normal cycle list:
        \pgfplotslistsize\autoplotspeclist\to\c@pgf@countd
        \pgfplots@getautoplotspec@{\autoplotspeclist}{\c@pgf@countd}{\pgfplots@listindex}%
        \let#1=\pgfplotsretval
    \fi
    \pgfmath@smuggleone#1%
    \endgroup
}

\pgfplotsset{
    cycle list set/.initial=
}
\makeatother
%%%%%%%%%%%%%%%%%%%%%%%%%%%%%%%%%%%%%%%%%%%%%%%%%%%%%%%%%%%%%%%%%%%%%%%%

\pgfplotsset{
/pgfplots/bar cycle list/.style={/pgfplots/cycle list={%
cyan!60!black,fill=cyan!80!black,mark=none*\\%
lime!80!black,fill=lime,mark=none*\\%
red,densely dashed,fill=red!80!black,mark=none*\\%
yellow!60!black,densely dashed,solid,fill=yellow!80!black,mark=none*\\%
black,solid,fill=gray,mark=none*\\%
red,densely dashed,fill=red!80!black,mark=mark=none*\\%
}
},
}
\pgfplotsset{soldot/.style={color=black,only marks,mark=*},
             holdot/.style={color=black,fill=white,only marks,mark=*},
             compat=1.12}

% Style to select only points from #1 to #2 (inclusive)
\pgfplotsset{select coords between index/.style 2 args={
    x filter/.code={
        \ifnum\coordindex<#1\def\pgfmathresult{}\fi
        \ifnum\coordindex>#2\def\pgfmathresult{}\fi
    }
}}

\pgfdeclarelayer{bg}    % declare background layer
\pgfsetlayers{bg,main}  % set the order of the layers (main is the standard layer

\tikzset{>=latex}

% \newbox{\LegDas}
% \savebox{\LegDas}{
%     (\begin{tikzpicture}
%     \draw[black,thick,dashdotted](0,0.09) -- (0.5,0.09);
%     \phantom{\draw[white](0,0) -- (0.09,0);}
%     \end{tikzpicture})}

% \newbox{\LegDash}
% \savebox{\LegDash}{
%     (\begin{tikzpicture}
%     \draw[black,thick,dashed](0,0.09) -- (0.55,0.09);
%     \phantom{\draw[white](0,0) -- (0.09,0);}
%     \end{tikzpicture})}
