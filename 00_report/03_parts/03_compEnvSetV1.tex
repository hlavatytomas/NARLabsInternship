\cleardoublepage
\section{Computational environment}
\label{sec:env}

NCHC servers use Singularity container platform~\cite{singularity} and Slurm workload manager~\cite{slurm} to run user-defined tasks. Thus,  
\begin{enumerate}[(i)]
    \item custom singularity container, which includes necessary applications and packages installed inside, is prepared, and
    \item slurm task (which works within the container) is prepared and run at the computational server.
\end{enumerate}

\subsection{Preparation of the singularity container}
\label{subsec:prepCont}

Assuming you have super-user permission and singularity installed, a preparation of the singularity container image from docker ubuntu:latest release and with openfoam.org/v10 and other useful applications installed inside can be done as follows:
\begin{itemize}
    \item Navigate outside the home directory, e.g. \texttt{/tmp/test/} and work here.
    \item New container (\texttt{./ubuntu}) is created from ubuntu docker repository, \texttt{----sandbox} flag allows to write into it later:\\[0.2cm] 
    \texttt{sudo singularity build ----sandbox ./ubuntu docker://ubuntu:latest}
    \item Shell inside container is opened with \texttt{----writable} flag to install necessary stuff into container:\\[0.2cm] 
    \texttt{sudo singularity shell ./ubuntu ----writable}
    \item Installation of the basic applications and openfoam.org/v10 into container:\\[0.2cm] 
    \texttt{apt update}\\
    \texttt{apt install python3 python3-pip wget vim software-properties-common \\ \indent\quad\quad python3-tk}\\
    \texttt{pip3 install matplotlib}\\
    \texttt{sh -c "wget -O - https://dl.openfoam.org/gpg.key >} \\ \indent\quad\quad\texttt{/etc/apt/trusted.gpg.d/openfoam.asc"}\\
    \texttt{add-apt-repository http://dl.openfoam.org/ubuntu}\\
    \texttt{apt update}\\
    \texttt{apt install openfoam10}
    \item If you want to compile custom openfoam solver, source openfoam in container:\\[0.2cm] 
    \texttt{. /opt/openfoam10/etc/bashrc}
    \item and compile it using \texttt{wmake} in prepared solver directory (in our case in\\ \texttt{./01\_codes/OF\_cases/03\_customSolvers/pollutionFoam}).
    \item When everything is installed, the \texttt{.sif} container file can be build using:\\[0.2cm] 
    \texttt{sudo singularity build ubuntu.sif ./ubuntu/}
\end{itemize}

Following the above listed guideline, singularity container image \texttt{ubuntu.sif} is created. This can be uploaded to NCHC servers and used as described in following subsection. 

\subsection{Preparation of slurm control script and running of task}
\label{subsec:}
