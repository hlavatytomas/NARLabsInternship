\cleardoublepage
\section{Computational environment}
\label{sec:env}

NCHC servers use Singularity container platform~\cite{singularity} and Slurm workload manager~\cite{slurm} to run user-defined tasks. Thus,  
\begin{enumerate}[(i)]
    \item custom singularity container, which includes necessary packages installed and compiled inside, is prepared, and
    \item Slurm task (which runs within the container) is prepared and run at the computational server.
\end{enumerate}

\subsection{Preparation of the singularity container}
\label{subsec:prepCont}

Assuming you have super-user permission and singularity installed on local (see step-by-step guide in \cite{singularityInstall}), a preparation of the singularity container image from docker ubuntu:latest release can be done as follows: 
\begin{enumerate}
    \item Navigate outside the home directory and work here, e.g.: \\[0.2cm] 
    \texttt{mkdir /tmp/} \\[0.2cm] 
    \texttt{mkdir /tmp/test} \\[0.2cm] 
    \texttt{cd /tmp/test/} 
    \item New container (\texttt{./ubuntu}) can be built from ubuntu docker repository using:\\[0.2cm] 
    \texttt{sudo singularity build ----sandbox ./ubuntu docker://ubuntu:latest}\\[0.2cm] 
    {NOTE:} \texttt{----sandbox} flag allows to write into container later
\end{enumerate}

Shell inside container can be opened using: \\[0.2cm]
\indent\quad\quad\texttt{sudo singularity shell ./ubuntu ----writable} \\[0.2cm]
where \texttt{----writable} flag again allows to write into container and install packages here.

Openfoam.org/v10 and other used packages can be installed inside the container as:\\[0.2cm]
    \indent\quad\quad\texttt{apt update}\\[0.2cm]
    \indent\quad\quad\texttt{apt install python3 python3-pip wget vim software-properties-common \\ \indent\quad\quad\quad\quad python3-tk}
    \indent\quad\quad\texttt{pip3 install matplotlib}\\[0.2cm]
    \indent\quad\quad\texttt{sh -c "wget -O - https://dl.openfoam.org/gpg.key >} \\ \indent\quad\quad\quad\quad\texttt{/etc/apt/trusted.gpg.d/openfoam.asc"}\\[0.2cm]
    \indent\quad\quad\texttt{add-apt-repository http://dl.openfoam.org/ubuntu}\\[0.2cm]
    \indent\quad\quad\texttt{apt update}\\[0.2cm]
    \indent\quad\quad\texttt{apt install openfoam10}

Compilation of the custom solver inside container is done as follows:
\begin{enumerate}
    \item Source OpenFOAM in the container shell:\\[0.2cm] 
    \indent\quad\quad\texttt{. /opt/openfoam10/etc/bashrc}
    \item Navigate to solver folder, e.g.:\\[0.2cm] 
    \indent\quad\quad\texttt{cd /tmp/pollutionFoam} 
    \item Compile solver executing: \\[0.2cm] 
    \indent\quad\quad\texttt{wmake}.
\end{enumerate}

When everything is installed, the \texttt{.sif} container file can be built from prepared \texttt{/tmp/test/ubuntu} directory using:\\[0.2cm] 
\indent\quad\quad\texttt{sudo singularity build /tmp/test/ubuntu.sif /tmp/test/ubuntu/}


Following the above listed guideline, singularity container image \texttt{ubuntu.sif} is created. This can be uploaded to NCHC servers and used as described in following subsection. 

\subsection{Preparation of Slurm control script and running the task}
\label{subsec:slurmCtrl}

