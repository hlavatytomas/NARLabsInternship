\cleardoublepage
\pagenumbering{arabic}
\section{Introduction}
\label{sec:intro}

The present work arose as the part of the authors student internship in Taiwan National Applied Research Laboratories (NARLabs). The report was written in National Center for High-performance Computing (NCHC). 

The main goal of the present work is the development of the OpenFOAM computational framework compatible with the NCHC computational servers. As the tutorial case of the interest, a spreading of an air pollution in an inhabited area has been chosen as it is one of the major environmental problems that affects human health and well-being. It is caused by various sources such as fossil fuel combustion, industrial emissions, agricultural activities, and natural phenomena~\cite{chen14}. Exposure to air pollutants can lead to respiratory diseases, cardiovascular diseases, cancer, and premature death~\cite{cohen17}. Therefore, it is essential to monitor and control air quality in different regions and settings, and to develop tools that can help with understanding of the spreading of the pollution.

The present work is structured as follows, 
\begin{inparaenum}[(i)]
    \item mathematical model is described together with the used assumptions (Section~\ref{sec:mathMod}), 
    \item computational framework (environment) is introduced with the usage guideline (Section~\ref{sec:env}), and
    \item the tutorial case results are briefly discussed (Section~\ref{sec:exp}).
\end{inparaenum}

